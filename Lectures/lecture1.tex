\section{Probabilities and Events}
\textbf{Date:} \underline{Aug 27, 2025}

\begin{definition}[Experiment]
A process that results in uncertain outcomes.
\end{definition}
\begin{definition}[Outcome]
A possible result of the experiment.
\end{definition}
\begin{definition}[Sample Space]
The set of all possible basic outcomes.
\end{definition}

\subsection{Events}
An \textbf{event} is a set of basic outcomes, i.e., a subset of the sample space.
\begin{itemize}
    \item Example 1: $\{\text{Tails}\}$
    \item Example 2: $\{\text{Admitted}\}$
    \item Example 3: $\{\text{Dog, Cat}\}$
\end{itemize}

\subsection{Probabilities}
\begin{definition}[Probability]
A numerical measure of the chance that an event will occur.
\end{definition}
\begin{itemize}
    \item $\Pr(E) = 1$ $\implies$ Event $E$ is certain to occur.
    \item $\Pr(E) = 0$ $\implies$ Event $E$ cannot occur.
    \item $\Pr(E) = 0.5$ $\implies$ Event $E$ is equally likely to occur or not.
\end{itemize}

\subsection{Interpretations of Probability}
\paragraph{Relative Frequency Viewpoint}
\[
    \Pr(E) = \lim_{n \to \infty} \frac{\text{Number of times } E \text{ occurs}}{n}
\]
Probabilities represent long-run relative frequencies when an experiment is repeated independently.

\paragraph{Subjective Probability}
Probability can also represent an individual’s degree of belief about an event.
\begin{itemize}
    \item Example: ``Will Lisa Cook be a Fed Governor on October 1?''
    \item Example: ``Which city is further west: Reno or Las Vegas?''
\end{itemize}

\subsection{Probability Distributions}
\begin{definition}[Probability Distribution]
A table assigning probabilities to all basic outcomes of an experiment, where the probabilities sum to $1$.
\end{definition}

\paragraph{Example: Preferred Pet}
\[
\begin{array}{c|c}
\text{Pet} & \Pr(\text{Pet}) \\ \hline
\text{Dog} & 0.44 \\
\text{Cat} & 0.30 \\
\text{Other} & 0.07 \\
\text{None} & 0.19 \\ \hline
\textbf{Total} & 1.00
\end{array}
\]

\subsection{Probability Tables}
When analyzing multiple attributes (e.g., gender and pet preference), we summarize data in a \textbf{probability table}.

\paragraph{Rules:}
\begin{enumerate}
    \item Two attributes: one on rows, one on columns.
    \item Events are mutually exclusive and exhaustive.
    \item Joint probabilities go in cells; marginal probabilities go in row and column totals.
\end{enumerate}

\paragraph{Example Table: Preferred Pet vs. Gender}
\[
\begin{array}{c|cccc|c}
 & \text{Dog} & \text{Cat} & \text{Other} & \text{None} & \Pr(\text{Gender}) \\ \hline
\text{Female} & 0.18 & 0.12 & 0.04 & 0.05 & 0.38 \\
\text{Male} & 0.32 & 0.15 & 0.05 & 0.09 & 0.62 \\
\text{Refuse/Other} & 0.00 & 0.00 & 0.00 & 0.00 & 0.00 \\ \hline
\Pr(\text{Pet}) & 0.50 & 0.27 & 0.09 & 0.14 & 1.00
\end{array}
\]

\subsection{Set Theory and Venn Diagrams}
Events can be represented as sets within a sample space.
\begin{itemize}
    \item $A \cap B$ = Event that both $A$ and $B$ occur.
    \item $A \cup B$ = Event that $A$ or $B$ (or both) occur.
    \item $A^c$ = Complement of $A$, i.e., $A$ does not occur.
\end{itemize}

\subsection{Three Useful Probability Rules}
\paragraph{Rule 1: Intersection}
\[
    \Pr(A \cap B) = \Pr(A \mid B) \Pr(B)
\]
\paragraph{Rule 2: Union}
\[
    \Pr(A \cup B) = \Pr(A) + \Pr(B) - \Pr(A \cap B)
\]
\paragraph{Rule 3: Complement}
\[
    \Pr(A^c) = 1 - \Pr(A)
\]

\subsection{Conditional Probability}
\begin{definition}[Conditional Probability]
The probability of event $A$ given event $B$:
\[
    \Pr(A \mid B) = \frac{\Pr(A \cap B)}{\Pr(B)}, \quad \Pr(B) > 0
\]
\end{definition}

\paragraph{Multiplication Rule}
\[
    \Pr(A \cap B) = \Pr(A \mid B)\Pr(B)
\]
