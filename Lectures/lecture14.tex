\section{Estimation}
\textbf{Date:} \underline{Nov 3, 2025}

\subsection{Overview}
\begin{itemize}
    \item \textbf{Big Picture}: We use random variables to represent uncertain outcomes (e.g., stock returns, household income).
    \item \textbf{Problem}: Typically, we don't know the probability distribution of interest (e.g., unknown expected value or variance).
    \item \textbf{Solution}: We estimate these properties with data by computing statistics that are good approximations to the unknowns.
\end{itemize}

\subsection{Samples and Populations}
\begin{itemize}
    \item We collect a \textbf{sample} of representative outcomes.
    \item We view the sample as a representative subset of some \textbf{population} (the whole class of individuals we want to characterize).
    \item Two examples:
    \begin{center}
    \begin{tabular}{c|c|c}
        & \textbf{Population} & \textbf{Sample} \\
        \hline
        Pet survey & Cornell students & 125 students who responded \\
        Daily stock returns & Past and future realizations & Past realizations \\
    \end{tabular}
    \end{center}
    \item When working with samples, we need:
    \begin{enumerate}
        \item A clear definition of the population.
        \item A reasonable probability model of how the sample was generated.
    \end{enumerate}
    \item \textbf{Comparison}:
    \begin{center}
    \begin{tabular}{c|c}
        \textbf{Population (Unknowns)} & \textbf{Sample (Observed)} \\
        \hline
        Random Variable $Y$ & Data $\{y_1, y_2, \ldots, y_N\}$ \\
        Unknown distribution & Sample histogram \\
        Mean $\mu = E[Y]$ & Sample average $\bar{y}$ \\
        Variance $\sigma^2 = E[(Y-\mu)^2]$ & Sample variance $s^2$ \\
    \end{tabular}
    \end{center}
    \item \textbf{Goal}: Statistical Inference --- Saying something about the population using the sample data.
    \item \textbf{Problems with Sampling}:
    \begin{enumerate}
        \item \textbf{Uncertainty}: Because you only have part of the population.
        \item \textbf{Bias}: Potential biases in the sample (e.g., media/poll bias).
    \end{enumerate}
\end{itemize}

\subsection{Estimating Means, Variances, Covariances, Correlations}
\begin{itemize}
    \item Population targets:
    \[
        \mu = E[Y], \qquad Var(Y)=E[(Y-E[Y])^2], \qquad Cov(X,Y)=E[(X-E[X])(Y-E[Y])].
    \]
    \item \textbf{Sample average} (estimator of $E[Y]$):
    \[
        \overline{y} = \frac{1}{n} \sum_{i=1}^{n} y_i = \frac{1}{n}(y_1 + y_2 + ... + y_n)
    \]
    \item \textbf{Sample variance} (estimator of $Var(Y)$):
    \[
        s^2 = \frac{1}{n-1} \sum_{i=1}^{n} (y_i - \overline{y})^2
    \]
    \item \textbf{Sample covariance} (estimator of $Cov(X,Y)$):
    \[
        s_{xy} = \frac{1}{n-1} \sum_{i=1}^{n} (x_i - \overline{x})(y_i - \overline{y})
    \]
    \item \textbf{Sample correlation}:
    \[
        r = \frac{s_{xy}}{s_x s_y}
    \]
    \item \textbf{Note}: The $n-1$ denominator is used for the same reason in both variance and covariance estimation.
\end{itemize}

\subsection{Example: Portfolio Analysis}
\begin{itemize}
    \item Consider holding a mix of two assets: "Experts' portfolio" and "Eli Lilly stock".
    \item Does holding a mix reduce risk?
    \item Given:
    \begin{itemize}
        \item Experts: $Var = 522.16$
        \item Eli Lilly: $Var = 327$
        \item $Cov(\text{Experts}, \text{Eli}) = 103.54$
        \item Weights: $w_E = 0.65$, $w_L = 0.35$
    \end{itemize}
    \item Variance of a two-asset mix (generic weights $w$ and $1-w$):
    \[
        \begin{aligned}
            Var(w\cdot \text{Eli} + (1-w)\cdot \text{Experts})
            &= w^2 Var(\text{Eli}) + (1-w)^2 Var(\text{Experts}) \\
            &\quad + 2w(1-w)Cov(\text{Eli}, \text{Experts})
        \end{aligned}
    \]
    \item Variance of the mix:
    \[
        Var(Mix) = (0.65)^2 (522.16) + (0.35)^2 (327) + 2(0.65)(0.35)(103.54) = 307.65
    \]
    \item Expected return of the mix:
    \[
        E[Mix] = 0.65(10.31) + 0.35(6.37) = 8.931
    \]
    \item Correlation between Experts and Eli Lilly:
    \[
        \mathrm{Corr}(\text{Experts}, \text{Eli}) \approx 0.25
    \]
    \item \textbf{Bottom Line}: A mixed portfolio often has lower variance than either component, even when they are positively correlated.
    \item \textbf{Financial Lessons}:
    \begin{enumerate}
        \item Optimal holdings depend on risk-return preferences.
        \item Some portfolios are sub-optimal for any investor.
    \end{enumerate}
\end{itemize}

\subsection{Quality of Estimates}
\begin{itemize}
    \item How good are our estimates?
    \item Let $\overline{X}$ be a random variable centered on the population mean.
    \item How do we get the standard deviation of $\overline{X}$?
    \begin{itemize}
        \item \textbf{Method 1}: Resample many times (Expensive).
        \item \textbf{Method 2}: Rely on Central Limit Theorem (CLT) since sample means are approximately normally distributed when $n$ is large (Cheap).
    \end{itemize}
    \item \textbf{Standard Error (SE)}: The standard deviation of an estimator (like $\overline{X}$), estimated by plugging in sample estimates for unknown parameters (like $\sigma$).
    \[
        SE(\overline{X}) = \frac{s}{\sqrt{n}}
    \]
    \item \textbf{Example (Birthweight)}:
    \begin{itemize}
        \item Sample A ($n=100, s=24$): $SE(\overline{X}) = s/\sqrt{n} = 24/10 = 2.4$.
        \item Sample B ($n=400, s=24$): $SE(\overline{X}) = s/\sqrt{n} = 24/20 = 1.2$.
    \end{itemize}
\end{itemize}
