\section{Randomized Experiments}
\textbf{Date:} \underline{Dec 3, 2025}

\subsection{Advertising Effectiveness (Motivation)}
\begin{itemize}
    \item A common observational setup (e.g., comScore/HBR-style studies) compares outcomes across consumers who:
    \begin{enumerate}
        \item saw no ads (control),
        \item saw only online display ads,
        \item saw only targeted search ads,
        \item saw both display and search ads,
    \end{enumerate}
    and then measures outcomes like website visits and (online/offline) sales.
    \item Key issue: these group comparisons may reflect selection, not causal effects.
\end{itemize}

\subsection{Notation and Definitions}
We are interested in the causal effect of a treatment (e.g., advertising).
\begin{itemize}
    \item $Y_{0i}$: Potential outcome for individual $i$ if \textbf{not} treated.
    \item $Y_{1i}$: Potential outcome for individual $i$ if treated.
    \item $D_i \in \{0, 1\}$: Indicator variable. $D_i = 1$ if treated, $0$ otherwise.
    \item \textbf{Observed Outcome} $Y_i$:
    \[ Y_i = (1 - D_i) Y_{0i} + D_i Y_{1i} = Y_{0i} + (Y_{1i} - Y_{0i})D_{i} \]
\end{itemize}

\subsection{Treatment Effects}
\begin{itemize}
    \item \textbf{Average Treatment Effect (ATE)}: The expected difference in outcomes if the entire population were treated vs. not treated.
    \[ \text{ATE} = E[Y_{1i} - Y_{0i}] \]
    \item \textbf{Treatment on the Treated (TT)}: The effect for those who actually received the treatment.
    \[ \text{TT} = E[Y_{1i} - Y_{0i} | D_i = 1] = E[Y_{1i} | D_i = 1] - E[Y_{0i} | D_i = 1] \]
    \item \textbf{Heterogeneous Treatment Effects}: Treatment effects $(Y_{1i}-Y_{0i})$ may differ across individuals (e.g., some groups respond more than others).
\end{itemize}

\subsection{Stylized Example}
\begin{itemize}
    \item Suppose the potential outcomes are:
    \begin{center}
    \begin{tabular}{c|c|c}
        $i$ & $Y_{0i}$ & $Y_{1i}$ \\
        \hline
        1 & 50 & 50 \\
        2 & 40 & 40 \\
        3 & 100 & 110 \\
        4 & 150 & 160 \\
    \end{tabular}
    \end{center}
    \item Estimated ATE:
    \[
        \widehat{ATE} = \frac{1}{4}\sum_{i=1}^{4}(Y_{1i}-Y_{0i}) = \frac{0+0+10+10}{4} = 5.
    \]
    \item If we observe $D_3=D_4=1$ and $D_1=D_2=0$, then the ``HBR effect'' (confounding comparison)
    \[
        E[Y_i\mid D_i=1]-E[Y_i\mid D_i=0] = \frac{110+160}{2}-\frac{50+40}{2} = 90.
    \]
\end{itemize}

\subsection{Selection Bias}
When we compare the average outcomes of treated vs. control groups using observational data, we get:
\[ \text{Observed Difference} = E[Y_i | D_i = 1] - E[Y_i | D_i = 0] \]
This can be decomposed as:
\[ = \underbrace{(E[Y_{1i} | D_i = 1] - E[Y_{0i} | D_i = 1])}_{\text{Treatment on the Treated}} + \underbrace{(E[Y_{0i} | D_i = 1] - E[Y_{0i} | D_i = 0])}_{\text{Selection Bias}} \]
\begin{itemize}
    \item \textbf{Selection Bias}: The difference in baseline outcomes (without treatment) between those who were treated and those who were not.
    \item Example: People who click on ads might be more likely to buy the product anyway (even without the ad).
\end{itemize}

\subsection{Randomization}
If treatment $D_i$ is randomly assigned, it is independent of potential outcomes $Y_{0i}, Y_{1i}$.
\begin{itemize}
    \item Independence implies Selection Bias is zero: $E[Y_{0i} | D_i = 1] = E[Y_{0i} | D_i = 0]$.
    \item Therefore, Observed Difference = ATE.
\end{itemize}

\subsection{Intent-to-Treat (ITT)}
What if not everyone assigned to the treatment group actually completes the treatment (non-compliance)?
\begin{itemize}
    \item \textbf{ITT Effect}: The difference between the outcome of the group \textit{assigned} to treatment and the group \textit{assigned} to control.
    \item This preserves the benefits of randomization even if compliance is imperfect.
\end{itemize}

\subsection{Case Studies}
\subsubsection{Online Advertising}
Comparing sales between those who saw ads and those who didn't often yields a "confounding comparison" due to selection bias. Randomized field experiments (like allocating cookies/users to treatment/control) are needed to measure the true lift.

\subsubsection{Breza et al. (2021): Social Media Messages}
\begin{itemize}
    \item \textbf{Research Question}: Can social media messages from health professionals influence behavior (holiday travel) and health outcomes (COVID-19 infections)?
    \item \textbf{Experiment}: A large-scale randomized control trial (RCT) using Facebook ads in the US before Thanksgiving and Christmas 2020.
    \item \textbf{Intervention}: Short video messages from doctors and nurses.
    \begin{itemize}
        \item Quote: ``This Thanksgiving, the best way to show your love is to stay home. If you do visit, wear a mask at all times... don't risk spreading COVID. Stay safe, stay home.''
    \end{itemize}
\end{itemize}

\paragraph{Randomization Design}
The study covered 820 counties across 13 states and used a two-stage randomization:
\begin{enumerate}
    \item \textbf{Stage 1 (County Level)}: Counties were randomized into ``high intensity'' (75\% treated zip codes) or ``low intensity'' (25\% treated zip codes) groups.
    \item \textbf{Stage 2 (Zip-code Level)}: Within counties, zip codes were randomized to Treatment (users see ads) or Control (no ads).
\end{enumerate}
This design allows for measuring both direct effects and spillovers (though the primary analysis focused on the direct intent-to-treat effect).

\paragraph{Results}
\begin{itemize}
    \item \textbf{First Outcome (Travel)}: Checked mobile phone location data.
    \begin{itemize}
        \item Result: Average distance traveled decreased by approx. 1 percentage point ($0.993$ ppt) in high-intensity counties compared to low-intensity counties in the 3 days before holidays.
    \end{itemize}
    \item \textbf{Second Outcome (COVID-19)}: Infections measured 2 weeks post-holiday.
    \begin{itemize}
        \item Result: ZIP codes with high-intensity treatment saw a \textbf{3.5\% reduction} in COVID-19 infections compared to controls.
    \end{itemize}
\end{itemize}

\paragraph{Conclusion}
Social media interventions by trusted professionals can be an effective low-cost public health tool. The study demonstrates the power of large-scale RCTs in digital platforms.
